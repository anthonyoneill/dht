\subsubsection{Valves}

A valve has a discharge relationship defined by 
\begin{align}
Q = C_d A_v \sqrt{2 g \Delta H},
\end{align}
where $C_d$ is the discharge coefficient and $A_v$ is the valve opening area. Therefore 
\begin{align}
\Delta H = \frac{Q^2}{2g\left(C_d A_v \right)^2} = \frac{k Q^2}{2 g A^2} \implies \frac{k Q^2}{2A} - g A \Delta H = 0,
\end{align}
where $A$ is the cross-sectional area and $k$ is the loss coefficient which varies depending upon the percentage opening of the valve. This means that the resistance term for valves may be written as
\begin{align} \label{valve_resistance}
\boxed{ R_j = - \frac{k_j Q_j|Q_j| }{2 A_j} + g A_j \Delta H_j. }
\end{align}
The loss coefficient $k$ may be determined from the discharge coefficient and the valve opening area using the relation 
\begin{align}
k = \frac{A^2}{\left(C_d A_v \right)^2}.
\end{align}

For example suppose we have a single valve resistance element with $H_0 = 20$, $H_1 = P_{atm} / \rho g$ with $k = 7$ and diameter $D = 50$mm. Then 
\begin{align}
Q =& \sqrt{\frac{2gA^2 \Delta H}{k}} = \pi D^2 \sqrt{\frac{g \left(H_0 - H_1 \right)}{8k}} \nonumber \\ =& \pi \cdot 0.05^2 \cdot \sqrt{\frac{9.80665 \cdot \left(20 - \left( \frac{101325}{ 997.0 * 9.80665} \right) \right)}{56}} \approx 0.010203.
\end{align} 
This solution agrees with the solution found from iteratively solving the linear system of equations. 

{\color{red} TODO pressure loss coefficient $k$ vs valve open percentage $\phi$ in a table.}

