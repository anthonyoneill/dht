\subsubsection{T junctions}

Consider the T junction shown in figure \ref{fig:t_junction_diagram}. The main pipe between $a$ and\
$b$ has a branch $c$ connected at 90°. The flow rates through each part of the junction, $Q_a$, \ 
$Q_b$ and $Q_c$, are considered positive towards the internal junction node $i$. 

% Diagram of T junction
\begin{figure}
    \centering
    \begin{tikzpicture}[scale=1, every node/.style={scale=0.8}] 
        \node[anchor=east] at (2.92,0) {$H_a$};
        \draw[fill=black] (3,0) circle (0.05cm);
        \node[anchor=south] at (4,0.08) {$Q_a$};
        \draw[thick] (4,0) node[cross=4pt] {};
        \node[anchor=south] at (5,0.08) {$H_i$};
        \draw[fill=black] (5,0) circle (0.05cm);
        \node[anchor=south] at (6,0.08) {$Q_b$};
        \draw[thick] (6,0) node[cross=4pt] {};
        \node[anchor=west] at (7.08,0) {$H_b$};
        \draw[fill=black] (7,0) circle (0.05cm);
        \draw[thick] (3,0) -- (5,0);
        \draw[thick] (5,0) -- (7,0);
        \node[anchor=north] at (5,-2.08) {$H_c$};
        \draw[fill=black] (5,-2) circle (0.05cm);
        \draw[thick] (5,0) -- (5,-2);
        \node[anchor=west] at (5.08,-1) {$Q_c$};
        \draw[thick] (5,-1) node[cross=4pt] {};
    \end{tikzpicture} 
    \caption{A T-junction is a three way pipe junction with a branch $c$ connected at 90° to 
    \ the main pipe between $a$ and $b$.}
    \label{fig:t_junction_diagram}
\end{figure}

The resistance function for each edge in the T junction is dependent upon the flow rates in the \
other two edges. Generally the resistance function for each edge is of the form
\begin{align}\label{eq:t_junction_resistance}
    \boxed{ R_j = - \frac{K_j Q_j |Q_j|}{2 A_j} + g A_j \Delta H_j, \hspace{0.5cm} j \in \{a,b,c\}, }
\end{align}
where the loss coefficients $K_a$, $K_b$ and $K_c$ are functions of the flow rates $Q_a$, $Q_b$ and \
$Q_c$. The form of the loss coefficients depends on the flow regime in the junction. The flow regime \
is determined by the sign of the flow rate in each edge \cite{rennels22}.

\paragraph{Diverging flow on main branch}

Flow diverging from the main branch is shown in figure \ref{fig:t_junction_diverging_main} where \ 
\ref{fig:t_junction_diverging_main_a} shows the flow diverging from node $a$ and \ 
\ref{fig:t_junction_diverging_main_b} shows the flow diverging from node $b$. The loss coefficient \ 
for the main branch is given by
\begin{align} \label{K_main_div}
    K_{\text{main,div}}(Q_{\text{in}}, Q_{\text{out}}) = 0.62 - 0.98 \gamma + 0.36 \gamma^2 + \
    0.03 \gamma^{-6},
\end{align}
where $\gamma = Q_{\text{in}} / Q_{\text{out}}$. For flow diverging from node $a$ we have \ 
$Q_{\text{in}} = Q_a$ and $Q_{\text{out}} = Q_b$ and for flow diverging from node $b$ we have \ 
$Q_{\text{in}} = Q_b$ and $Q_{\text{out}} = Q_a$.

For the side branch the loss coefficient is given by
\begin{align} \label{K_side_div}
    K_{\text{side,div}}(Q_{\text{in}}, Q_c) = \left(0.81-1.13\beta+\beta^2 \right)\hat{d}^4 \
    + 1.12 \hat{d} - 1.08\hat{d}^3 + K_{*},
\end{align}
where $\beta = Q_{\text{in}} / Q_c$ and $\hat{d} = d_c / d_{\text{main}}$. The constant\
$K_{*}$ is given by
\begin{align}
    K_{*} = 0.57 - 1.07 r_c^{\frac{1}{2}} -2.13 r_c + 8.24 r_c^{\frac{3}{2}} - 8.48 r_c^2 + \
    2.90 r_c^{\frac{5}{2}},
\end{align}
where $r_c = r / d_c$. Here $r$ is the radius of curvature of the junction and $d_c$ is the diameter\
of the side branch. For flow diverging from node $a$ we have $Q_{\text{in}} = Q_a$ and for flow \
diverging from node $b$ we have $Q_{\text{in}} = Q_b$.

\begin{figure}[h]
    \hspace{0.05\textwidth}
    \begin{subfigure}[b]{0.4\textwidth}
        \centering
        \begin{tikzpicture}[scale=0.5, every node/.style={scale=0.8}] 
            \node[anchor=east] at (2.92,0) {$a$};
            \draw[fill=black] (3,0) circle (0.05cm);
            \draw[fill=black] (5,0) circle (0.05cm);
            \node[anchor=west] at (7.08,0) {$b$};
            \draw[fill=black] (7,0) circle (0.05cm);
            \draw[thick] (3,0) -- (5,0);
            \draw[thick] (5,0) -- (7,0);
            \node[anchor=north] at (5,-2.08) {$c$};
            \draw[fill=black] (5,-2) circle (0.05cm);
            \draw[thick] (5,0) -- (5,-2);
            \draw[thick, ->] (5,-1) -- (5,-1.01); % c arrow
            \draw[thick, ->] (4,0) -- (4.01,0); % a arrow
            \draw[thick, ->] (6,0) -- (6.01,0); % b arrow
        \end{tikzpicture}
        \caption{Flow diverging from node $a$ has flow rates $Q_a > 0$, $Q_b < 0$ and $Q_c < 0$.}
        \label{fig:t_junction_diverging_main_a}
    \end{subfigure}
    \hspace{0.1\textwidth}
    \begin{subfigure}[b]{0.4\textwidth}
        \centering
        \begin{tikzpicture}[scale=0.5, every node/.style={scale=0.8}] 
            \node[anchor=east] at (2.92,0) {$a$};
            \draw[fill=black] (3,0) circle (0.05cm);
            \draw[fill=black] (5,0) circle (0.05cm);
            \node[anchor=west] at (7.08,0) {$b$};
            \draw[fill=black] (7,0) circle (0.05cm);
            \draw[thick] (3,0) -- (5,0);
            \draw[thick] (5,0) -- (7,0);
            \node[anchor=north] at (5,-2.08) {$c$};
            \draw[fill=black] (5,-2) circle (0.05cm);
            \draw[thick] (5,0) -- (5,-2);
            \draw[thick, ->] (5,-1) -- (5,-1.01); % c arrow
            \draw[thick, ->] (4,0) -- (3.99,0); % a arrow
            \draw[thick, ->] (6,0) -- (5.99,0); % b arrow
        \end{tikzpicture}
        \caption{Flow diverging from node $b$ has flow rates $Q_a < 0$, $Q_b > 0$ and $Q_c < 0$.}
        \label{fig:t_junction_diverging_main_b}
    \end{subfigure}
    \caption{Diverging flow on the main branch of a T junction.}
    \label{fig:t_junction_diverging_main}
\end{figure}

\paragraph{Converging flow on main branch}

Flow convering to the main branch is shown in figure \ref{fig:t_junction_converging_main} where \ 
\ref{fig:t_junction_converging_main_a} shows flow converging to node $a$ and \ 
\ref{fig:t_junction_converging_main_b} shows flow converging to node $b$. The loss coefficient \ 
for the main branch is given by
\begin{align} \label{eq:K_main_conv}
    K_{\text{main,conv}}(Q_{\text{in}}, Q_{\text{out}}) = \gamma^2 - 0.95 - \ 
    2C_{xC} \left(\gamma - 1 \right) - 2C_M \gamma ( \gamma - 1 ), 
\end{align}
where $\gamma = Q_{\text{in}} / Q_{\text{out}}$ and $C_{xC}$ and $C_M$ are given by
\begin{align}
    C_{M} &= 0.23 + 1.46 r_c -2.75 r_c^2 + 1.65 r_c^3 , \\
    C_{xC} &= 0.08 + 0.56 r_c -1.75 r_c^2 + 1.83 r_c^3.
\end{align}
For flow converging to node $a$ we have $Q_{\text{in}} = Q_b$ and $Q_{\text{out}}=Q_a$\
and for flow converging to node $b$ we have $Q_{\text{in}} = Q_a$ and $Q_{\text{out}}=Q_b$. 

For the side branch the loss coefficient is given by
\begin{align} \label{eq:K_side_conv}
    K_{\text{side,conv}}(Q_{\text{in}}, Q_c) = 2 C_{yC} - 1 + \
    2\left[\left(C_{xC}-1\right) + \left(2-C_{xC}-C_M \right) \beta - 0.46\beta^2 \right]\hat{d}^4,
\end{align}
where $\beta = Q_{\text{in}} / Q_c$ and $C_{yC}$ is given by
\begin{align}
    C_{yC} &= 1 - 0.25\hat{d}^{1.3} - \left[ 0.11 r_c -0.65 r_c^2 + 0.83 r_c^3 \right] \hat{d}^2.
\end{align}


\begin{figure}[h]
    \hspace{0.05\textwidth}
    \begin{subfigure}[b]{0.4\textwidth}
        \centering
        \begin{tikzpicture}[scale=0.5, every node/.style={scale=0.8}] 
            \node[anchor=east] at (2.92,0) {$a$};
            \draw[fill=black] (3,0) circle (0.05cm);
            \draw[fill=black] (5,0) circle (0.05cm);
            \node[anchor=west] at (7.08,0) {$b$};
            \draw[fill=black] (7,0) circle (0.05cm);
            \draw[thick] (3,0) -- (5,0);
            \draw[thick] (5,0) -- (7,0);
            \node[anchor=north] at (5,-2.08) {$c$};
            \draw[fill=black] (5,-2) circle (0.05cm);
            \draw[thick] (5,0) -- (5,-2);
            \draw[thick, ->] (5,-1) -- (5,-0.99); % c arrow
            \draw[thick, ->] (4,0) -- (3.99,0); % a arrow
            \draw[thick, ->] (6,0) -- (5.99,0); % b arrow
        \end{tikzpicture}
        \caption{Flow converging to node $a$ has flow rates $Q_a < 0$, $Q_b > 0$ and $Q_c > 0$.}
        \label{fig:t_junction_converging_main_a}
    \end{subfigure}
    \hspace{0.1\textwidth}
    \begin{subfigure}[b]{0.4\textwidth}
        \centering
        \begin{tikzpicture}[scale=0.5, every node/.style={scale=0.8}] 
            \node[anchor=east] at (2.92,0) {$a$};
            \draw[fill=black] (3,0) circle (0.05cm);
            \draw[fill=black] (5,0) circle (0.05cm);
            \node[anchor=west] at (7.08,0) {$b$};
            \draw[fill=black] (7,0) circle (0.05cm);
            \draw[thick] (3,0) -- (5,0);
            \draw[thick] (5,0) -- (7,0);
            \node[anchor=north] at (5,-2.08) {$c$};
            \draw[fill=black] (5,-2) circle (0.05cm);
            \draw[thick] (5,0) -- (5,-2);
            \draw[thick, ->] (5,-1) -- (5,-0.99); % c arrow
            \draw[thick, ->] (4,0) -- (4.01,0); % a arrow
            \draw[thick, ->] (6,0) -- (6.01,0); % b arrow
        \end{tikzpicture}
        \caption{Flow converging to node $b$ has flow rates $Q_a > 0$, $Q_b < 0$ and $Q_c > 0$.}
        \label{fig:t_junction_converging_main_b}
    \end{subfigure}
    \caption{Converging flow on the main branch of a T junction.}
    \label{fig:t_junction_converging_main}
\end{figure}

\paragraph{Converging or diverging flow from the side branch}

Flow on the side branch is shown in figure \ref{fig:t_junction_side} where\ 
\ref{fig:t_junction_converging_side} shows flow converging to the side branch and\
\ref{fig:t_junction_diverging_side} shows flow diverging from the side branch.\
For flow converging to the side branch we have the loss coefficient 
\begin{align} \label{eq:K_conv_to_side}
    K_{\text{convto}}(Q_{\text{in}}, Q_{\text{out}}) = \  
    \left(0.81 - 1.16 \sqrt{r_c} + 0.5 r_c \right) \gamma^2 \ 
    - \left( 0.95 - 1.65 r_c \right) \gamma + 1.34 - 1.69 r_c,
\end{align}
where $\gamma = Q_{\text{in}} / Q_{\text{out}} = Q_a / Q_c$.\
For flow diverging from the side branch we have the loss coefficient 
\begin{align} \label{eq:K_div_from_side}
    K_{\text{divfrom}}(Q_{\text{in}}, Q_{\text{out}}) = 0.59 \gamma^2 \ 
    + \left( 1.18 - 1.84 \sqrt{r_c} + 1.16 r_c \right) \gamma \ 
    - 0.68 + 1.04 \sqrt{r_c} - 1.16 r_c,
\end{align}
where $\gamma = Q_{\text{in}} / Q_{\text{out}} = Q_c / Q_a$.


\begin{figure}[h]
    \hspace{0.05\textwidth}
    \begin{subfigure}[b]{0.4\textwidth}
        \centering
        \begin{tikzpicture}[scale=0.5, every node/.style={scale=0.8}] 
            \node[anchor=east] at (2.92,0) {$a$};
            \draw[fill=black] (3,0) circle (0.05cm);
            \draw[fill=black] (5,0) circle (0.05cm);
            \node[anchor=west] at (7.08,0) {$b$};
            \draw[fill=black] (7,0) circle (0.05cm);
            \draw[thick] (3,0) -- (5,0);
            \draw[thick] (5,0) -- (7,0);
            \node[anchor=north] at (5,-2.08) {$c$};
            \draw[fill=black] (5,-2) circle (0.05cm);
            \draw[thick] (5,0) -- (5,-2);
            \draw[thick, ->] (5,-1) -- (5,-1.01); % c arrow
            \draw[thick, ->] (4,0) -- (4.01,0); % a arrow
            \draw[thick, ->] (6,0) -- (5.99,0); % b arrow
        \end{tikzpicture}
        \caption{Flow converging to node $c$ has flow rates $Q_a > 0$, $Q_b > 0$ and $Q_c < 0$.}
        \label{fig:t_junction_converging_side}
    \end{subfigure}
    \hspace{0.1\textwidth}
    \begin{subfigure}[b]{0.4\textwidth}
        \centering
        \begin{tikzpicture}[scale=0.5, every node/.style={scale=0.8}] 
            \node[anchor=east] at (2.92,0) {$a$};
            \draw[fill=black] (3,0) circle (0.05cm);
            \draw[fill=black] (5,0) circle (0.05cm);
            \node[anchor=west] at (7.08,0) {$b$};
            \draw[fill=black] (7,0) circle (0.05cm);
            \draw[thick] (3,0) -- (5,0);
            \draw[thick] (5,0) -- (7,0);
            \node[anchor=north] at (5,-2.08) {$c$};
            \draw[fill=black] (5,-2) circle (0.05cm);
            \draw[thick] (5,0) -- (5,-2);
            \draw[thick, ->] (5,-1) -- (5,-0.99); % c arrow
            \draw[thick, ->] (4,0) -- (3.99,0); % a arrow
            \draw[thick, ->] (6,0) -- (6.01,0); % b arrow
        \end{tikzpicture}
        \caption{Flow diverging from node $c$ has flow rates $Q_a < 0$, $Q_b < 0$ and $Q_c > 0$.}
        \label{fig:t_junction_diverging_side}
    \end{subfigure}
    \caption{Converging/diverging flow to the side branch of a T junction.}
    \label{fig:t_junction_side}
\end{figure}

\paragraph{Loss coefficients}

For each of the six possible flow regimes the loss coefficient in each edge, $K_a$, $K_b$ and\
$K_c$, takes on a different value. The loss coefficient $K_a$ is given by 
\begin{align} \label{eq:K_a}
    \boxed{ K_a(Q_a, Q_b, Q_c) = \begin{cases}
        0 & \text{if } Q_a > 0, Q_b < 0, Q_c < 0, \\
        K_{\text{main,div}}(Q_b, Q_a) & \text{if } Q_a < 0, Q_b > 0, Q_c < 0, \\
        0 & \text{if } Q_a < 0, Q_b > 0, Q_c > 0, \\
        K_{\text{main,conv}}(Q_a, Q_b) & \text{if } Q_a > 0, Q_b < 0, Q_c > 0, \\
        K_{\text{convto}}(Q_a, Q_c) & \text{if } Q_a > 0, Q_b > 0, Q_c < 0, \\
        K_{\text{divfrom}}(Q_c, Q_a) & \text{if } Q_a < 0, Q_b < 0, Q_c > 0,
    \end{cases} }
\end{align}
$K_b$ is given by
\begin{align} \label{eq:K_b}
    \boxed{ K_b(Q_a, Q_b, Q_c) = \begin{cases}
        K_{\text{main,div}}(Q_a, Q_b) & \text{if } Q_a > 0, Q_b < 0, Q_c < 0, \\
        0 & \text{if } Q_a < 0, Q_b > 0, Q_c < 0, \\
        K_{\text{main,conv}}(Q_b, Q_a) & \text{if } Q_a < 0, Q_b > 0, Q_c > 0, \\
        0 & \text{if } Q_a > 0, Q_b < 0, Q_c > 0, \\
        K_{\text{convto}}(Q_a, Q_c) & \text{if } Q_a > 0, Q_b > 0, Q_c < 0, \\
        K_{\text{divfrom}}(Q_c, Q_a) & \text{if } Q_a < 0, Q_b < 0, Q_c > 0,
    \end{cases} }
\end{align}
and $K_c$ is given by
\begin{align} \label{eq:K_c}
    \boxed{ K_c(Q_a, Q_b, Q_c) = \begin{cases}
        K_{\text{side,div}}(Q_a, Q_c) & \text{if } Q_a > 0, Q_b < 0, Q_c < 0, \\
        K_{\text{side,div}}(Q_b, Q_c) & \text{if } Q_a < 0, Q_b > 0, Q_c < 0, \\
        K_{\text{side,conv}}(Q_b, Q_c) & \text{if } Q_a < 0, Q_b > 0, Q_c > 0, \\
        K_{\text{side,conv}}(Q_a, Q_c) & \text{if } Q_a > 0, Q_b < 0, Q_c > 0, \\
        0 & \text{if } Q_a > 0, Q_b > 0, Q_c < 0, \\
        0 & \text{if } Q_a < 0, Q_b < 0, Q_c > 0.
    \end{cases} }
\end{align}
Substiting these loss coefficients into the function \eqref{eq:t_junction_resistance} we are able \ 
to calculate the resistance of the T junction for any flow regime.

{\color{red} TODO modify Jacobian due to cross derivative terms?}

% https://www.mathworks.com/help/hydro/ref/tjunctiontl.html


