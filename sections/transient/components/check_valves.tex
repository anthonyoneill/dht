\subsubsection{Check / Non-return valves}

A check or non-return valve is a valve that allows fluid to flow through it in one direction only. \ 
The resistance of a check valve is given by
\begin{align} \label{check_valve_resistance}
    \boxed{ R_j = - \frac{Q_j|Q_j| }{2 A_j} + k_j^{-1} g A_j \Delta H_j, }
\end{align}
where the transient coefficient in the matrix $\mathbf{B}$ is
\begin{align}
    \boxed{ B_{j,j} = k_j^{-1} }
\end{align}
and the inverse loss coefficient $k_j^{-1}$ is given by
\begin{align}
    \boxed{ 
        k_j^{-1} = 
        \begin{cases}
            0 & \text{if } \Delta H_j \leq 0 \\
            k_0^{-1} & \text{if } \Delta H_j > 0.
        \end{cases}
    }
\end{align}
This means that the check valve is open when the pressure difference across it is positive and \ 
closed when the pressure difference is negative. The value $k_0$ is the loss coefficient of the \ 
valve when it is fully open. For steady flow, the check valve is considered to be open \ 
$(\tau = 1)$, so the inverse loss coefficient is $k_0^{-1}$. 
